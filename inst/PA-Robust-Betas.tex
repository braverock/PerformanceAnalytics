%% LyX 2.3.7 created this file.  For more info, see http://www.lyx.org/.
%% Do not edit unless you really know what you are doing.
\documentclass[11pt]{extarticle}\usepackage[]{graphicx}\usepackage[]{xcolor}
% maxwidth is the original width if it is less than linewidth
% otherwise use linewidth (to make sure the graphics do not exceed the margin)
\makeatletter
\def\maxwidth{ %
  \ifdim\Gin@nat@width>\linewidth
    \linewidth
  \else
    \Gin@nat@width
  \fi
}
\makeatother

\definecolor{fgcolor}{rgb}{0.345, 0.345, 0.345}
\newcommand{\hlnum}[1]{\textcolor[rgb]{0.686,0.059,0.569}{#1}}%
\newcommand{\hlstr}[1]{\textcolor[rgb]{0.192,0.494,0.8}{#1}}%
\newcommand{\hlcom}[1]{\textcolor[rgb]{0.678,0.584,0.686}{\textit{#1}}}%
\newcommand{\hlopt}[1]{\textcolor[rgb]{0,0,0}{#1}}%
\newcommand{\hlstd}[1]{\textcolor[rgb]{0.345,0.345,0.345}{#1}}%
\newcommand{\hlkwa}[1]{\textcolor[rgb]{0.161,0.373,0.58}{\textbf{#1}}}%
\newcommand{\hlkwb}[1]{\textcolor[rgb]{0.69,0.353,0.396}{#1}}%
\newcommand{\hlkwc}[1]{\textcolor[rgb]{0.333,0.667,0.333}{#1}}%
\newcommand{\hlkwd}[1]{\textcolor[rgb]{0.737,0.353,0.396}{\textbf{#1}}}%
\let\hlipl\hlkwb

\usepackage{framed}
\makeatletter
\newenvironment{kframe}{%
 \def\at@end@of@kframe{}%
 \ifinner\ifhmode%
  \def\at@end@of@kframe{\end{minipage}}%
  \begin{minipage}{\columnwidth}%
 \fi\fi%
 \def\FrameCommand##1{\hskip\@totalleftmargin \hskip-\fboxsep
 \colorbox{shadecolor}{##1}\hskip-\fboxsep
     % There is no \\@totalrightmargin, so:
     \hskip-\linewidth \hskip-\@totalleftmargin \hskip\columnwidth}%
 \MakeFramed {\advance\hsize-\width
   \@totalleftmargin\z@ \linewidth\hsize
   \@setminipage}}%
 {\par\unskip\endMakeFramed%
 \at@end@of@kframe}
\makeatother

\definecolor{shadecolor}{rgb}{.97, .97, .97}
\definecolor{messagecolor}{rgb}{0, 0, 0}
\definecolor{warningcolor}{rgb}{1, 0, 1}
\definecolor{errorcolor}{rgb}{1, 0, 0}
\newenvironment{knitrout}{}{} % an empty environment to be redefined in TeX

\usepackage{alltt}
\usepackage[T1]{fontenc}
\usepackage[latin9]{inputenc}
\usepackage{geometry}
\geometry{verbose,tmargin=1in,bmargin=1in,lmargin=1in,rmargin=1in}
\setlength{\parskip}{\medskipamount}
\setlength{\parindent}{0pt}
\usepackage{color}
\usepackage{float}
\usepackage{url}
\usepackage{amsmath}
\usepackage{amsthm}
\usepackage{graphicx}
\usepackage{setspace}
\onehalfspacing
\usepackage[unicode=true,pdfusetitle,
 bookmarks=true,bookmarksnumbered=true,bookmarksopen=false,
 breaklinks=false,pdfborder={0 0 1},backref=false,colorlinks=false]
 {hyperref}

\makeatletter
%%%%%%%%%%%%%%%%%%%%%%%%%%%%%% User specified LaTeX commands.
%%%%%%%%%%%%%%%%%%%% book.tex %%%%%%%%%%%%%%%%%%%%%%%%%%%%%
%
% sample root file for the chapters of your "monograph"
%
% Use this file as a template for your own input.
%
%%%%%%%%%%%%%%%% Springer-Verlag %%%%%%%%%%%%%%%%%%%%%%%%%%


% RECOMMENDED %%%%%%%%%%%%%%%%%%%%%%%%%%%%%%%%%%%%%%%%%%%%%%%%%%%


% choose options for [] as required from the list
% in the Reference Guide


\usepackage[bottom]{footmisc}% places footnotes at page bottom

\usepackage{hyperref}
\def\UrlBreaks{\do\/\do-}

\usepackage[style=authoryear,natbib=true,firstinits=true,backend=biber]{biblatex}

\setlength{\bibitemsep}{0.5\baselineskip}
\DeclareNameAlias{sortname}{last-first}

\addbibresource{robregJAM.bib}
\renewcommand\cite{\citet}

%\renewcommand*{\nameyeardelim}{\addspace}
%\renewbibmacro{in:}{}

\usepackage{txfonts}
\usepackage{upgreek}
\usepackage{booktabs}

\hypersetup{pdfstartview={XYZ null null 1.00}}

\makeatother
\IfFileExists{upquote.sty}{\usepackage{upquote}}{}
\begin{document}


\title{Robust Betas for PerformanceAnalytics}
\maketitle
\begin{center}
R. Douglas Martin and Dhairya Jain
\par\end{center}

\tableofcontents{}

\section{Introduction}

A time series Single Factor Model (SFM) has the general form

\begin{equation}
r_{t}=\alpha+\beta f_{t}+\epsilon_{t}\,t=1,2,\cdots,T\label{eq: singleFactorModel}
\end{equation}

where $r_{t}$ is a time series of asset returns, such as those for
a stock, an ETF or a hedge fund, $f_{t}$ is a factor return, $\epsilon_{t}$
is the error term, and $\alpha$ and $\beta$ are unknown intercept
and slope coefficients that need to be estimated based on observed
asset and factor returns. The factor return $f_{t}$ is typically
either: (1) a market proxy $f_{M,t}$, such as the CRSP value-weighted
market index $f_{M,t}=f_{CRSP,t}$ in the context of the Capital Asset
Pricing Model (CAPM), or (2) an active manager's index benchmark such
as the S\&P500, Russell 1000, Russell 2000, and Russell 3000, among
others.

The SFM errors for arbitrary intercept $a$ and slope $b$ values
are defined as:

\begin{equation}
\epsilon_{t}(a,b)=r_{t}-a-bf_{t},\,t=1,2,\cdots,T.\label{eq: errors}
\end{equation}

For estimates $a=\hat{\alpha}$ and $b=\hat{\beta}$, the above errors
become the fitted model \emph{residuals}

\begin{equation}
\hat{\epsilon}_{t}=\epsilon_{t}(\hat{\alpha},\hat{\beta})=r_{t}-\hat{\alpha}-\hat{\beta}f_{t}\label{eq: residuals}
\end{equation}

and one has the fit-plus-residuals representation of the asset return:

\begin{equation}
r_{t}=\hat{\alpha}-\hat{\beta}f_{t}+\hat{\epsilon}_{t}.\label{eq: fitPlusResiduals}
\end{equation}

The unknown coefficients $\alpha$ and $\beta$ are currently almost
universally estimated using the method of least squares (LS), which
is computed by minimizing the sum of the squared errors $\sum_{t=1}^{T}\epsilon_{t}^{2}(a,b)$
with respect to $a$ and $b$. Unfortunately, both asset returns and
factor returns often contain outliers which adversely influence the
LS intercept and slope estimates, and one needs a robust alternative
which is computed as a complement to LS, or a replacement for LS,
depending on the context.

In this Vignette we describe and illustrate the use of a highly robust
estimator of SFM slope and intercept parameters. This estimator called
the \emph{mOpt} estimator (the Robust estimator), and it has an intuitive
weighted-least-squares (WLS) interpretation that it minimizes the
weighted sum of squared regression residuals $\sum_{t=1}^{T}w_{t}(a,b)\epsilon_{t}^{2}(a,b)$,
where $w_{t}(a,b)$ is a special data-dependent weight function described
in Section \ref{sec: mOpEstimatorDetails}.

The \emph{mOpt} estimator was recently introduced for time series
factor models in \citet{MartinXiaTS2022}, who used it to study the
performance of mOpt relative to LS in estimating CAPM betas for the
cross-section of liquid U.S. stocks from 1963 to 2018, and for fitting
multifactor time series models such as the Fama-French 3 factor model.\footnote{This paper appeared in the March 2022 issue of The Journal of Asset
Management, and is available in opens source form at \url{https://link.springer.com/content/pdf/10.1057/s41260-022-00258-0.pdf}.} The study revealed extensive adverse influence of outliers on the
LS betas. In particular, it was shown that the LS and mOpt betas differ
in absolute values by at least 0.3 for roughly 26\% of microcap stocks,
14\% of smallcaps and 7\% of bigcaps, and by at least 0.5 for roughly
12\% of microcap stocks, 5\% of smallcaps and 2\% of bigcaps.

Section 2 introduces the Robust SFM fitting functions in the PerformanceAnalytics
R package, and illustrates their use for package \texttt{managers}
data set. Section 3 provides some mathematical details for the mOpt
Robust estimator, and Section 4 provides concluding comments.

\section{The PerformanceAnalytics Robust Betas Functions}

PerformanceAnalytics package contains the following two main functions
for computing LS and mOpt (Robust) SFM model fits:
\begin{itemize}
\item \texttt{chart.SFM}
\item \texttt{SFM.fit.models}
\end{itemize}
The \texttt{chart.SFM} function computes both LS and Robust alphas
and betas and overlays the corresponding straight line fits to a scatter
plot of asset returns versus benchmark or market proxy returns. The
\texttt{SFM.fit.models} function also computes both LS and Robust
SFM fits, in order to provide: (a) a comparative tabular display of
the LS and Robust alphas, betas, and related statistics, and (b) an
optional display of any subset of 10 different comparative graphical
displays of the LS and Robust SFM fits.

In addition the following function allows the user to compute either
mOpt or LS robust SFM fits, with mOpt the default, for any combination
of one or more sets of asset returns and one or more benchmarks:
\begin{itemize}
\item \texttt{SFM.coefficients}
\end{itemize}
In order to use these functions, an R user needs to first install
the current version of PerformanceAnalytics from CRAN (\url{https://cran.r-project.org/web/packages/PerformanceAnalytics/index.html}),
and load it with:

\begin{knitrout}
\definecolor{shadecolor}{rgb}{0.969, 0.969, 0.969}\color{fgcolor}\begin{kframe}
\begin{alltt}
\hlkwd{library}\hlstd{(PerformanceAnalytics)}
\end{alltt}
\end{kframe}
\end{knitrout}



\subsection{The \texttt{managers} Data Set}

The following examples will use the \texttt{xts} time series data
set \texttt{managers} included with PerformanceAnalytics, so we first
load this data set and determine its class, dimensions, and names
with the code:

\begin{knitrout}
\definecolor{shadecolor}{rgb}{0.969, 0.969, 0.969}\color{fgcolor}\begin{kframe}
\begin{alltt}
\hlkwd{data}\hlstd{(managers)}
\hlkwd{class}\hlstd{(managers)}
\end{alltt}
\begin{verbatim}
## [1] "xts" "zoo"
\end{verbatim}
\begin{alltt}
\hlkwd{dim}\hlstd{(managers)}
\end{alltt}
\begin{verbatim}
## [1] 132  10
\end{verbatim}
\begin{alltt}
\hlkwd{names}\hlstd{(managers)}
\end{alltt}
\begin{verbatim}
##  [1] "HAM1"        "HAM2"        "HAM3"        "HAM4"        "HAM5"       
##  [6] "HAM6"        "EDHEC LS EQ" "SP500 TR"    "US 10Y TR"   "US 3m TR"
\end{verbatim}
\end{kframe}
\end{knitrout}

The results show that \texttt{managers} is an \texttt{xts} data object
consisting of 10 time series for 132 months. Now we replace the last
4 names of \texttt{managers} with shorter convenient names with no
spaces:



Next we use the function \texttt{tsPlotMP} from the PCRA package to
create the time series plots shown in Figure \ref{fig: mgrsTseries}.

\begin{knitrout}
\definecolor{shadecolor}{rgb}{0.969, 0.969, 0.969}\color{fgcolor}\begin{kframe}
\begin{alltt}

\hlstd{PCRA}\hlopt{::}\hlkwd{tsPlotMP}\hlstd{(managers,} \hlkwc{scaleType} \hlstd{=} \hlstr{"same"}\hlstd{,} \hlkwc{axis.cex} \hlstd{=} \hlnum{0.5}\hlstd{,} \hlkwc{stripText.cex} \hlstd{=} \hlnum{0.5}\hlstd{)}
\end{alltt}
\begin{verbatim}
\end{verbatim}
\end{kframe}
\end{knitrout}

\begin{figure}[H]
\begin{centering}
\includegraphics[width=1\textwidth]{Plots/mgrsTseries.png}
\par\end{centering}
\caption{Time Series of managers Data}

\label{fig: mgrsTseries}
\end{figure}

The figure shows that some of the times series data begins in January
1996, but some series begin later, and all series continue until December
2008. It will be convenient for the calculations below to use the
maximum time window of the managers data such that none of the time
series have missing data. This maximum length window is easily determine
using the \texttt{na.omit} and \texttt{range} functions as follows:

\begin{knitrout}
\definecolor{shadecolor}{rgb}{0.969, 0.969, 0.969}\color{fgcolor}\begin{kframe}
\begin{alltt}
\hlkwd{range}\hlstd{(}\hlkwd{index}\hlstd{(}\hlkwd{na.omit}\hlstd{(managers)))}
\end{alltt}
\begin{verbatim}
## [1] "2001-09-30" "2006-12-31"
\end{verbatim}
\end{kframe}
\end{knitrout}

In order to avoid using last four months fraction of the year 2001,
we use the following code line to delete those four months and rename
the result:

\begin{knitrout}
\definecolor{shadecolor}{rgb}{0.969, 0.969, 0.969}\color{fgcolor}\begin{kframe}
\begin{alltt}
\hlstd{mgrs} \hlkwb{<-} \hlstd{managers[}\hlstr{"2002/"}\hlstd{]}  \hlcom{# 5 full years from 2002 through 2006}
\end{alltt}
\end{kframe}
\end{knitrout}

\subsection{Use of \texttt{chart.SFM}}

The function \texttt{chart.SFM} computes LS and mOpt robust alphas
and betas, and makes a graphical display of the resulting LS and mOpt
straight line fits, superimposed on the scatter plot of asset returns
and factor returns. The arguments of \texttt{chart.SFM} are viewed
with:

\begin{knitrout}
\definecolor{shadecolor}{rgb}{0.969, 0.969, 0.969}\color{fgcolor}\begin{kframe}
\begin{alltt}
\hlkwd{args}\hlstd{(chart.SFM)}
\end{alltt}
\begin{verbatim}
## function (Ra, Rb, Rf = 0, main = NULL, ylim = NULL, xlim = NULL, 
##     family = "mopt", xlab = NULL, ylab = NULL, legend.loc = "topleft", 
##     makePct = FALSE) 
## NULL
\end{verbatim}
\end{kframe}
\end{knitrout}

NOTE: With the default NULL optional arguments for the \texttt{xlim}
and \texttt{ylim} axes limits, \texttt{chart.SFM} uses sensible data
dependent values, as will be seen in resulting plot in the figure
below. You can obtain more information about the arguments of \texttt{chart.SFM}
by using the \texttt{help()} function:

\begin{knitrout}
\definecolor{shadecolor}{rgb}{0.969, 0.969, 0.969}\color{fgcolor}\begin{kframe}
\begin{alltt}
\hlkwd{help}\hlstd{(chart.SFM)}
\end{alltt}
\end{kframe}
\end{knitrout}

Figure \ref{fig: mOptLSforHAM6} shows the result of using \texttt{chart.SFM}
for the HAM6 returns and the S\&P500 as the benchmark.

\begin{knitrout}
\definecolor{shadecolor}{rgb}{0.969, 0.969, 0.969}\color{fgcolor}\begin{kframe}
\begin{alltt}
\hlkwd{chart.SFM}\hlstd{(mgrs}\hlopt{$}\hlstd{HAM6, mgrs}\hlopt{$}\hlstd{SP500, mgrs}\hlopt{$}\hlstd{RF, }\hlkwc{makePct} \hlstd{=} \hlnum{TRUE}\hlstd{)}
\end{alltt}
\begin{verbatim}
\end{verbatim}
\end{kframe}
\end{knitrout}

\begin{figure}[H]
\begin{centering}
\includegraphics[width=0.67\textwidth]{Plots/mOptLSforHAM6.png}
\par\end{centering}
\caption{Robust mOpt Beta and LS Beta for HAM6}

\label{fig: mOptLSforHAM6}
\end{figure}

You can use a simple \texttt{for} loop to plot the mOpt and LS fits
for any subset of the HAM1 through HAM6 funds and LSEQ, for example
you do so for HAM3 through HAM6 with the code line:

\begin{knitrout}
\definecolor{shadecolor}{rgb}{0.969, 0.969, 0.969}\color{fgcolor}\begin{kframe}
\begin{alltt}
\hlkwa{for}\hlstd{(k} \hlkwa{in} \hlnum{3}\hlopt{:}\hlnum{6}\hlstd{)\{}
    \hlkwd{chart.SFM}\hlstd{(mgrs[,k], mgrs}\hlopt{$}\hlstd{SP500, mgrs}\hlopt{$}\hlstd{RF, }\hlkwc{makePct} \hlstd{= T,}
        \hlkwc{main} \hlstd{=} \hlkwd{names}\hlstd{(mgrs[,k]))}
\hlstd{\}}
\end{alltt}
\end{kframe}
\end{knitrout}

\begin{figure}[H]
\begin{centering}
\includegraphics[width=0.5\textwidth]{Plots/HAM3.png}\includegraphics[width=0.5\textwidth]{Plots/HAM4.png}
\par\end{centering}
\begin{centering}
\includegraphics[width=0.5\textwidth]{Plots/HAM5.png}\includegraphics[width=0.5\textwidth]{Plots/HAM6.png}
\par\end{centering}
\caption{Robust mOpt Beta and LS Beta for HAM3 - HAM6}

\label{fig: mOptLSforHAM3-6}
\end{figure}


\subsection{Use of \texttt{SFM.fit.models}}

The function \texttt{SFM.fit.models} has two main capabilities. The
first is to compute both LS and Robust alpha and beta coefficient
estimates, along with their standard errors and t-statistics, and
display them in easy-to-compare table form. The second is to make
side-by-side graphical displays of Robust and LS model fitting results.
First, we find out what the arguments of \texttt{SFM.fit.models} are
with:

\begin{knitrout}
\definecolor{shadecolor}{rgb}{0.969, 0.969, 0.969}\color{fgcolor}\begin{kframe}
\begin{alltt}
\hlkwd{args}\hlstd{(SFM.fit.models)}
\end{alltt}
\begin{verbatim}
## function (Ra, Rb, Rf = 0, family = "mopt", which.plots = NULL, 
##     plots = TRUE) 
## NULL
\end{verbatim}
\end{kframe}
\end{knitrout}

Use the first code line below to compute the Robust and LS coefficients
with no graphics, and save the result of the fitted models object
fitHAM6, displays. Then use the next 3 code lines to see what the
class of fitHAM6 is, display the LS and Robust alpha and beta coefficient
estimates, and display a complete comparative LS and Robust statistics
summary:

\begin{knitrout}
\definecolor{shadecolor}{rgb}{0.969, 0.969, 0.969}\color{fgcolor}\begin{kframe}
\begin{alltt}
\hlstd{fitHAM6} \hlkwb{<-} \hlkwd{SFM.fit.models}\hlstd{(mgrs}\hlopt{$}\hlstd{HAM6, mgrs}\hlopt{$}\hlstd{SP500, }\hlkwc{Rf} \hlstd{= mgrs}\hlopt{$}\hlstd{RF, }
    \hlkwc{plots} \hlstd{=} \hlnum{FALSE}\hlstd{)}
\hlkwd{class}\hlstd{(fitHAM6)}
\end{alltt}
\begin{verbatim}
## [1] "lmfm"       "fit.models"
\end{verbatim}
\begin{alltt}
\hlkwd{round}\hlstd{(}\hlkwd{coef}\hlstd{(fitHAM6),}\hlnum{3}\hlstd{)}
\end{alltt}
\begin{verbatim}
##        (Intercept)  Beta
## LSFit        0.006 0.325
## RobFit       0.007 0.548
\end{verbatim}
\begin{alltt}
\hlkwd{summary}\hlstd{(fitHAM6)}
\end{alltt}
\begin{verbatim}
## 
## Calls:
##  LSFit: lm(formula = Alpha ~ Beta, data = merged, subset = subset)
## RobFit: RobStatTM::lmrobdetMM(formula = Alpha ~ Beta, data = merged, 
##     subset = subset, control = RobStatTM::lmrobdet.control(family = family))
## 
## Residual Statistics:
##              Min       1Q   Median      3Q     Max
##  LSFit: -0.04504 -0.01249 0.003436 0.01277 0.04062
## RobFit: -0.06474 -0.01175 0.003893 0.01185 0.06478
## 
## Coefficients:
##                      Estimate Std. Error t value Pr(>|t|)    
## (Intercept):  LSFit: 0.006331   0.002632   2.405   0.0194 *  
##              RobFit: 0.006753   0.002392   2.824   0.0065 ** 
##                                                              
##        Beta:  LSFit: 0.325048   0.073839   4.402 4.67e-05 ***
##              RobFit: 0.547858   0.080665   6.792 6.55e-09 ***
## ---
## Signif. codes:  0 '***' 0.001 '**' 0.01 '*' 0.05 '.' 0.1 ' ' 1
## 
## Residual Scale Estimates:
##  LSFit: 0.02028 on 58 degrees of freedom
## RobFit: 0.01943 on 58 degrees of freedom
## 
## Multiple R-squared:
##  LSFit: 0.2504 
## RobFit: 0.2364
\end{verbatim}
\end{kframe}
\end{knitrout}

To make a wide variety of plots that compare Robust and LS SFM model
fitting results use

\begin{knitrout}
\definecolor{shadecolor}{rgb}{0.969, 0.969, 0.969}\color{fgcolor}\begin{kframe}
\begin{alltt}
\hlkwd{SFM.fit.models}\hlstd{(mgrs}\hlopt{$}\hlstd{HAM6,mgrs}\hlopt{$}\hlstd{SP500,}\hlkwc{Rf} \hlstd{= mgrs}\hlopt{$}\hlstd{RF)}
\end{alltt}
}\end{kframe}
\end{knitrout}

which results in the following output in the Console:

\texttt{\textcolor{blue}{Make plot selections (or 0 to exit): }}

\texttt{\textcolor{blue}{1: Normal QQ Plot of Residuals}}

\texttt{\textcolor{blue}{2: Kernel Density Estimate of Residuals}}

\texttt{\textcolor{blue}{3: Residuals vs. Mahalanobis Distance}}

\texttt{\textcolor{blue}{4: Residuals vs. Fitted Values}}

\texttt{\textcolor{blue}{5: Sqrt Residuals vs. Fitted Values}}

\texttt{\textcolor{blue}{6: Response vs. Fitted Values}}

\texttt{\textcolor{blue}{7: Residuals vs. Index (Time)}}

\texttt{\textcolor{blue}{8: Overlaid Normal QQ Plot of Residuals}}

\texttt{\textcolor{blue}{9: Overlaid Kernel Density Estimate of Residuals}}

\texttt{\textcolor{blue}{10: Scatter Plot with Overlaid Fit(s)}}

\texttt{\textcolor{blue}{Selection:}}

The first time you try this, we suggest that you enter each of the
choices 1 through 10 after Selection, and after each selection you
with see the corresponding plot type, then enter 0 to exit from the
graphics display menu. If you just want a particular subset of graphical
displays, e.g., types 2 and 7, just enter 2 to see the first plot
and then enter 7 after Selection, followed by 0 to Exit. Alternatively,
use of the command

\begin{knitrout}
\definecolor{shadecolor}{rgb}{0.969, 0.969, 0.969}\color{fgcolor}\begin{kframe}
\begin{alltt}
\hlkwd{SFM.fit.models}\hlstd{(mgrs}\hlopt{$}\hlstd{HAM6, mgrs}\hlopt{$}\hlstd{SP500, }\hlkwc{Rf} \hlstd{= mgrs}\hlopt{$}\hlstd{RF, }\hlkwc{which.plots} \hlstd{=} \hlkwd{c}\hlstd{(}\hlnum{2}\hlstd{,}\hlnum{7}\hlstd{))}\end{alltt}
\begin{verbatim}
\end{verbatim}
\end{kframe}
\end{knitrout}
results in the following in the Console:

\texttt{\textcolor{blue}{Hit <Return> to see next plot:}}

Then pressing Enter results in display of the type 2 plot in the top
of Figure \ref{fig:  plotTypes 2 and 7}, and pressing Enter again
results in display of the bottom 7 plot in Figure \ref{fig:  plotTypes 2 and 7},
and then the above line in the Console disappears. Try it out.

\begin{figure}[H]
\begin{centering}
\includegraphics[width=0.6\textwidth]{Plots/kernelDensityEstimates.png}
\par\end{centering}
\begin{centering}
\includegraphics[width=0.6\textwidth]{Plots/residualsVsTime.png}
\par\end{centering}
\caption{The Top Plot is Type 2 and the Bottom Plot is Type 7}

\label{fig:  plotTypes 2 and 7}
\end{figure}


\subsection{Use of SFM.coefficients}

The function \texttt{SFM.coefficients} was designed to support computing
either Robust mOpt (the default) or LS multiple single factor model
(SFM) for one or more assets and one or more benchmarks. Here are
the arguments of \texttt{SFM.coefficients}:

\begin{knitrout}
\definecolor{shadecolor}{rgb}{0.969, 0.969, 0.969}\color{fgcolor}\begin{kframe}
\begin{alltt}
\hlkwd{args}\hlstd{(SFM.coefficients)}
\end{alltt}
\begin{verbatim}
## function (Ra, Rb, Rf = 0, subset = TRUE, ..., method = "Robust", 
##     family = "mopt", digits = 3, benchmarkCols = T, Model = F, 
##     warning = T) 
## NULL
\end{verbatim}
\end{kframe}
\end{knitrout}

Here we use the first four managers HAM1, HAM2, HAM3, HAM4 as the
assets, and the SP500 and Bond10Yr as the benchmarks:

\begin{knitrout}
\definecolor{shadecolor}{rgb}{0.969, 0.969, 0.969}\color{fgcolor}\begin{kframe}
\begin{alltt}
\hlstd{funds} \hlkwb{<-} \hlstd{mgrs[, }\hlkwd{c}\hlstd{(}\hlstr{"HAM1"}\hlstd{, }\hlstr{"HAM2"}\hlstd{, }\hlstr{"HAM3"}\hlstd{, }\hlstr{"HAM4"}\hlstd{)]}
\hlstd{benchmarks} \hlkwb{<-} \hlstd{mgrs[, }\hlkwd{c}\hlstd{(}\hlstr{"SP500"}\hlstd{, }\hlstr{"Bond10Yr"}\hlstd{)]}
\end{alltt}
\end{kframe}
\end{knitrout}

Now we make Robust and LS fits of the four mangers to the two benchmarks,
and examine the class of the resulting \texttt{fit.Rob} (\texttt{fit.LS}
has the same matrix class).

\begin{knitrout}
\definecolor{shadecolor}{rgb}{0.969, 0.969, 0.969}\color{fgcolor}\begin{kframe}
\begin{alltt}
\hlstd{(fit.ROB} \hlkwb{<-} \hlkwd{SFM.coefficients}\hlstd{(funds, benchmarks,} \hlkwc{method} \hlstd{=}\hlstr{"Robust"}\hlstd{))}
\end{alltt}
\begin{verbatim}
##      Alpha : SP500 Alpha : Bond10Yr Beta : SP500 Beta : Bond10Yr
## HAM1         0.006            0.012        0.595          -0.334
## HAM2         0.001            0.002        0.184          -0.275
## HAM3         0.003            0.008        0.586          -0.331
## HAM4         0.003            0.017        1.190          -0.286
\end{verbatim}
\begin{alltt}
\hlstd{(fit.LS} \hlkwb{<-} \hlkwd{SFM.coefficients}\hlstd{(funds, benchmarks,} \hlkwc{method} \hlstd{=}\hlstr{"LS"}\hlstd{))}
\end{alltt}
\begin{verbatim}
##      Alpha : SP500 Alpha : Bond10Yr Beta : SP500 Beta : Bond10Yr
## HAM1         0.006            0.011        0.599          -0.426
## HAM2         0.002            0.004        0.216          -0.215
## HAM3         0.002            0.007        0.555          -0.378
## HAM4         0.008            0.015        0.923          -0.429
\end{verbatim}
\begin{alltt}
\hlkwd{class}\hlstd{(fit.ROB)}
\end{alltt}
\begin{verbatim}
## [1] "matrix" "array"
\end{verbatim}
\begin{alltt}
\hlkwd{summary}\hlstd{(fit.ROB)}
\end{alltt}
\begin{verbatim}
##  Alpha : SP500     Alpha : Bond10Yr   Beta : SP500    Beta : Bond10Yr  
##  Min.   :0.00100   Min.   :0.00200   Min.   :0.1840   Min.   :-0.3340  
##  1st Qu.:0.00250   1st Qu.:0.00650   1st Qu.:0.4855   1st Qu.:-0.3317  
##  Median :0.00300   Median :0.01000   Median :0.5905   Median :-0.3085  
##  Mean   :0.00325   Mean   :0.00975   Mean   :0.6388   Mean   :-0.3065  
##  3rd Qu.:0.00375   3rd Qu.:0.01325   3rd Qu.:0.7438   3rd Qu.:-0.2833  
##  Max.   :0.00600   Max.   :0.01700   Max.   :1.1900   Max.   :-0.2750
\end{verbatim}
\end{kframe}
\end{knitrout}

We note that since \texttt{method = ``Robust''} is the default,
that argument may be omitted in the first code line above.

By default, the benchmark Alpha and Beta results are in columns, and
those of the assets are in rows. This is because portfolio managers
often have many assets in their portfolio and only a few benchmarks.
Note that \texttt{fit.Rob} and \texttt{fit.LS} are R matrix objects,
and the results are printed with the default digits = 3. You can get
the robust fit results displayed with benchmarks in rows, and 6 significant
digits with the code line:

\begin{knitrout}
\definecolor{shadecolor}{rgb}{0.969, 0.969, 0.969}\color{fgcolor}\begin{kframe}
\begin{alltt}
\hlkwd{SFM.coefficients}\hlstd{(funds, benchmarks,} \hlkwc{benchmarkCols} \hlstd{=} \hlnum{FALSE}\hlstd{,} \hlkwc{digits} \hlstd{=} \hlnum{6}\hlstd{)}
\end{alltt}
\begin{verbatim}
##                       HAM1      HAM2      HAM3      HAM4
## Alpha : SP500     0.005602  0.001172  0.003275  0.002534
## Alpha : Bond10Yr  0.011842  0.002482  0.007719  0.017016
## Beta : SP500      0.594613  0.183761  0.586422  1.189505
## Beta : Bond10Yr  -0.334156 -0.274526 -0.330747 -0.285967
\end{verbatim}
\end{kframe}
\end{knitrout}

You can use the function \texttt{SFM.alpha} if you only want alpha
estimates, and use \texttt{SFM.beta} if you only want beta estimates.
For example the following gives robust alphas

\begin{knitrout}
\definecolor{shadecolor}{rgb}{0.969, 0.969, 0.969}\color{fgcolor}\begin{kframe}
\begin{alltt}
\hlkwd{SFM.alpha}\hlstd{(funds, benchmarks,} \hlkwc{digits} \hlstd{=} \hlnum{4}\hlstd{)}
\end{alltt}
\begin{verbatim}
##      Alpha : SP500 Alpha : Bond10Yr
## HAM1        0.0058           0.0110
## HAM2        0.0023           0.0044
## HAM3        0.0024           0.0071
## HAM4        0.0078           0.0148
\end{verbatim}
\end{kframe}
\end{knitrout}

and the following gives LS betas:

\begin{knitrout}
\definecolor{shadecolor}{rgb}{0.969, 0.969, 0.969}\color{fgcolor}\begin{kframe}
\begin{alltt}
\hlkwd{SFM.alpha}\hlstd{(funds, benchmarks,} \hlkwc{method} \hlstd{=} \hlstr{"LS"}\hlstd{,} \hlkwc{digits} \hlstd{=} \hlnum{2}\hlstd{)}
\end{alltt}
\begin{verbatim}
##      Alpha : SP500 Alpha : Bond10Yr
## HAM1          0.01             0.01
## HAM2          0.00             0.00
## HAM3          0.00             0.01
## HAM4          0.01             0.01
\end{verbatim}
\end{kframe}
\end{knitrout}

\section{The mOpt Robust SFM Fit Mathematical Details\label{sec: mOpEstimatorDetails}}

The SFM model \ref{eq: singleFactorModel} can be written in the following
form:

\begin{align*}
r_{t} & =\mathbf{\tilde{f}}_{t}^{\prime}\boldsymbol{\theta}+s\epsilon_{t},\;t=1,2,\cdots,T
\end{align*}

where $\mathbf{\tilde{f}}_{t}^{\prime}=(1,f_{t})$, $\boldsymbol{\theta}=(\alpha,\beta)^{\prime}$,
and $\epsilon_{t}$ is a standardized error term that is scaled by
the scale parameter $s$. The robust estimate $\hat{\boldsymbol{\theta}}=(\hat{\alpha},\hat{\beta})$
is a solution of the weighted least squares (WLS) estimating equation

\begin{equation}
\sum_{i=1}^{T}w_{t}\mathbf{\tilde{f}}_{t}\left(r_{t}-\mathbf{\tilde{f}}_{t}^{\prime}\hat{\boldsymbol{\theta}}\right)=0\label{eq: regMestWLSEqn}
\end{equation}

where the $w_{t}$ are the data--dependent weights

\begin{equation}
w_{t}=w_{t}(\hat{\boldsymbol{\theta}};\hat{s})=w_{mOpt}\left(\frac{r_{t}-\mathbf{\tilde{f}}_{t}^{\prime}\hat{\boldsymbol{\theta}}}{\hat{s}}\right)\label{eq: regMestWeights}
\end{equation}

and the shape of the weight function $w_{mOpt}(t)$ is shown in Figure
\ref{fig: mOptWtFunction}.

\begin{figure}[H]
\begin{centering}
\includegraphics[width=0.7\textwidth]{Plots/mOptWt.png}
\par\end{centering}
\caption{The mOpt weight function ($c=3.00$)}

\label{fig: mOptWtFunction}
\end{figure}

The mOpt weight function gives a weight of 1 to all sufficiently small
robustly scaled residuals $\hat{\epsilon}_{t}=(r_{t}-\mathbf{\tilde{f}}_{t}^{\prime}\hat{\boldsymbol{\theta}})/\hat{s}$,
and smoothly transitions to zero weight for robustly scaled residuals
whose absolute is greater than 3.00. All asset and factor returns
pairs $(r_{t},\mathbf{\tilde{f}}_{t})$ whose scaled residuals have
absolute values larger than 3.0 are are \emph{rejected} by the by
the $\hat{\boldsymbol{\theta}}$ estimator. For normally distributed
data and true parameter values, the probability that such a pair is
rejected is only 0.27\%, and the estimator is essentially equivalent
to the LS estimator.

The weights $w_{t}=w_{t}(\hat{\boldsymbol{\theta}};\hat{s})$ depend
on the values of $\hat{\boldsymbol{\theta}}$, $r_{t}$, and $\mathbf{\tilde{f}}_{t}$.
Consequently. the WLS equation (\ref{eq: regMestWLSEqn}) is a nonlinear
function of the data $(r_{t},f_{t}),t=1,\cdots,T$, and $\hat{\boldsymbol{\theta}}$
must be computed with some type of iterative nonlinear algorithm.
It is quite convenient that the estimate $\hat{\boldsymbol{\theta}}$
may be expressed in the nonlinear weighted least squares (WLS) mathematical
form

\begin{equation}
\hat{\boldsymbol{\theta}}=\left(\sum_{i=1}^{T}w_{t}(\hat{\boldsymbol{\theta}};\hat{s})\mathbf{\tilde{f}}_{t}\mathbf{\tilde{f}}_{t}^{\prime}\right)^{-1}\left(\sum_{i=1}^{T}w_{t}(\hat{\boldsymbol{\theta}};\hat{s})\mathbf{\tilde{f}}_{t}r_{t}\right)\label{eq: robregWLSversion-1}
\end{equation}

which lends itself to the iterated weighted least squares (IRWLS)
algorithm:

\begin{equation}
\hat{\boldsymbol{\theta}}^{k+1}=\left(\sum_{i=1}^{T}w_{t}(\hat{\boldsymbol{\theta}}^{k};\hat{s})\mathbf{\tilde{f}}_{t}\mathbf{\tilde{f}}_{t}^{\prime}\right)^{-1}\left(\sum_{i=1}^{T}w_{t}(\hat{\boldsymbol{\theta}}^{k};\hat{s})\mathbf{\tilde{f}}_{t}r_{t}\right),\,k=0,1,2,\cdots.\label{eq: regMestIRWLS-1}
\end{equation}

The mOpt Robust estimator is computed with the above IRWLS algorithm,
using a highly robust but inefficient initial estimate $\hat{\boldsymbol{\theta}}^{0}$.
For further details, see \citet{MartinXiaTS2022}.

\printbibliography
\end{document}
