%% no need for  \DeclareGraphicsExtensions{.pdf,.eps}

\documentclass[12pt,letterpaper,english]{article}
\usepackage{times}
\usepackage[T1]{fontenc}
\IfFileExists{url.sty}{\usepackage{url}}
                      {\newcommand{\url}{\texttt}}

\usepackage{babel}
%\usepackage{noweb}
\usepackage{Rd}

\usepackage{Sweave}

%\VignetteIndexEntry{Performance Attribution from Bacon}
%\VignetteDepends{PerformanceAnalytics}
%\VignetteKeywords{returns, performance, risk, benchmark, portfolio}
%\VignettePackage{PerformanceAnalytics}

%\documentclass[a4paper]{article}
%\usepackage[noae]{Sweave}
%\usepackage{ucs}
%\usepackage[utf8x]{inputenc}
%\usepackage{amsmath, amsthm, latexsym}
%\usepackage[top=3cm, bottom=3cm, left=2.5cm]{geometry}
%\usepackage{graphicx}
%\usepackage{graphicx, verbatim}
%\usepackage{ucs}
%\usepackage[utf8x]{inputenc}
%\usepackage{amsmath, amsthm, latexsym}
%\usepackage{graphicx}

\title{GLM Smoothing Index}
\author{R Project for Statistical Computing}

\begin{document}
\input{GLMSmoothIndex-concordance}

\maketitle


\begin{abstract}
The returns to hedge funds and other alternative investments are often highly serially correlated.Gemanstsy,Lo and Markov propose an econometric model of return smoothingand develop estimators for the smoothing profile.The magnitude of impact is measured by the smoothing index, which is a measure of concentration of weight in lagged terms.
\end{abstract}



\section{Background}
To quantify the impact of all of these possible sources of serial correlation, denote by \(R_t\),the true economic return of a hedge fund in period t; and let \(R_t\) satisfy the following linear single-factor model:

\begin{equation}
 R_t  =   \\ {\mu} + {\beta}{{\delta}}_t+ \xi_t
\end{equation}

Where $\xi_t,  \sim N(0,1)$
and Var[\(R_t\)] = $\sigma$\ \(^2\)

True returns represent the flow of information that would determine the equilibrium value of the fund's securities in a frictionless market. However, true economic returns are not observed. Instead, \(R_t^0\) denotes the reported or observed return in period t; and let
%$Z = \sin(X)$. $\sqrt{X}$.
  
%$\hat{\mu}$ = $\displaystyle\frac{22}{7}$
%e^{2 \mu} = 1
%\begin{equation}
%\left(\sum_{t=1}^{T} R_t/T\right) = \hat{\mu} \\
%\end{equation}
\begin{equation}
 R_t^0  = \theta _0R_{t} + \theta _1R_{t-1}+\theta _2R_{t-2}  + \cdots +  \theta _kR_{t-k}\\
\end{equation}
\begin{equation}
\theta _j \epsilon [0,1] where : j = 0,1, \cdots , k  \\
\end{equation}

and 
%\left(\mu \right) =  \sum_{t=1}^{T} \(Ri)/T\ \\
\begin{equation}
\theta _1 + \theta _2 + \theta _3 \cdots + \theta _k = 1  \\
\end{equation}

which is a weighted average of the fund's true returns over the most recent k + 1
periods, including the current period.

\section{Smoothing Index}
A useful summary statistic for measuringthe concentration of weights is :
\begin{equation}
\xi =   \sum_{j=0}^{k} \theta _j^2 \\
\end{equation}

This measure is well known in the industrial organization literature as the Herfindahl index, a measure of the concentration of firms in a given industry where $\theta$\(_j\) represents the market share of firm j. Becaus $\xi_t$\ is confined to the unit interval, and is minimized when all the $\theta$\(_j\) 's are identical, which implies a value of 1/k+1 for $\xi_i$\ ; and is maximized when one coefficient is 1 and the rest are 0. In the context of smoothed returns, a lower value of implies more smoothing, and the upper bound of 1 implies no smoothing, hence we shall refer to $\theta$\(_j\) as a ''\textbf{smoothingindex}''.

\section{Usage}

In this example we use edhec database, to compute Smoothing Index for Hedge Fund Returns.
\begin{Schunk}
\begin{Sinput}
> library(PerformanceAnalytics)
> data(edhec)
> GLMSmoothIndex(edhec)
\end{Sinput}
\begin{Soutput}
                 Convertible Arbitrage CTA Global Distressed Securities
GLM Smooth Index             0.3487825  0.1866095             0.3187229
                 Emerging Markets Equity Market Neutral Event Driven
GLM Smooth Index        0.3022908             0.2046973    0.3580198
                 Fixed Income Arbitrage Global Macro Long/Short Equity
GLM Smooth Index              0.3090088     0.252546          0.277132
                 Merger Arbitrage Relative Value Short Selling Funds of Funds
GLM Smooth Index        0.2292355      0.2917355     0.2348319      0.2873716
\end{Soutput}
\end{Schunk}


\end{document}
