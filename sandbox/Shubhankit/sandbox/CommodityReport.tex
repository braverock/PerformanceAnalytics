%% no need for  \DeclareGraphicsExtensions{.pdf,.eps}

\documentclass[12pt,letterpaper,english]{article}
\usepackage{times}
\usepackage[T1]{fontenc}
\IfFileExists{url.sty}{\usepackage{url}}
                      {\newcommand{\url}{\texttt}}

\usepackage{babel}
%\usepackage{noweb}
\usepackage{Rd}

\usepackage{Sweave}

%\VignetteIndexEntry{Performance Attribution from Bacon}
%\VignetteDepends{PerformanceAnalytics}
%\VignetteKeywords{returns, performance, risk, benchmark, portfolio}
%\VignettePackage{PerformanceAnalytics}

%\documentclass[a4paper]{article}
%\usepackage[noae]{Sweave}
%\usepackage{ucs}
%\usepackage[utf8x]{inputenc}
%\usepackage{amsmath, amsthm, latexsym}
%\usepackage[top=3cm, bottom=3cm, left=2.5cm]{geometry}
%\usepackage{graphicx}
%\usepackage{graphicx, verbatim}
%\usepackage{ucs}
%\usepackage[utf8x]{inputenc}
%\usepackage{amsmath, amsthm, latexsym}
%\usepackage{graphicx}

\title{Commodity Index Fund Performance Analysis}
\author{Shubhankit Mohan}

\begin{document}
\Sconcordance{concordance:CommodityReport.tex:CommodityReport.Rnw:%
1 46 1 1 6 28 1 1 7 1 2 6 1 1 2 20 0 1 2 3 1 1 2 15 0 1 2 16 1 1 7 1 2 %
12 1 1 7 1 2 3 1 1 4 1 2 9 1 1 2 7 0 1 1 9 0 1 2 8 1 1 6 1 2 11 1 1 3 1 %
2 2 1 1 3 1 2 7 1 2 2 2 1}


\maketitle


\begin{abstract}
The fact that many hedge fund returns exhibit extraordinary levels of serial correlation is now well-known and generally accepted as fact. The effect of this autocorrelation on investment returns diminishes the apparent risk of such asset classes as the true returns/risk is easily \textbf{camouflaged} within a haze of illiquidity, stale prices, averaged price quotes and smoothed return reporting. We highlight the effect \emph{autocorrelation} and \emph{drawdown} has on performance analysis by investigating the results of functions developed during the Google Summer of Code 2013 on \textbf{commodity based index} .
\end{abstract}

\tableofcontents



\section{Background}
The investigated fund index that tracks a basket of \emph{commodities} to measure their performance.The value of these indexes fluctuates based on their underlying commodities, and this value depends on the \emph{component}, \emph{methodology} and \emph{style} to cover commodity markets .

A brief overview of the four index invested in our report are : 
  \begin{itemize}
    \item
     \textbf{DJUBS Commodity index} :  is a broadly diversified index that allows investors to track commodity futures through a single, simple measure. As the index has grown in popularity since its introduction in 1998, additional versions and a full complement of subindices have been introduced. Together, the family offers investors a comprehensive set of tools for measuring the commodity markets.
      \item
 \textbf{Morningstar CLS index} : is a simple rules-based trend following index operated in commodities
   \item
    \textbf{Newedge CTI} :  includes funds that utilize a variety of investment strategies to profit from price moves in commodity markets.
Managers typically use either (i) a trading orientated approach,involving the trading of physical commodity products and/or of commodity
derivative instruments in either directional or relative value strategies; Or (ii) Long short equity strategies focused on commodity related stocks.
  \end{itemize}
%Let $X \sim N(0,1)$ and $Y \sim \textrm{Exponential}(\mu)$.  Let
%$Z = \sin(X)$. $\sqrt{X}$.
  
%$\hat{\mu}$ = $\displaystyle\frac{22}{7}$
%e^{2 \mu} = 1
%\begin{equation}
%\left(\sum_{t=1}^{T} R_t/T\right) = \hat{\mu} \\
%\end{equation}

\section{Performance Summary Chart}

Given a series  of historical returns \((R_1,R_2, . . .,R_T)\) from \textbf{January-2001} to \textbf{December-2009}, create a wealth index chart, bars for per-period performance, and underwater chart for drawdown of the 3 funds.

\includegraphics{CommodityReport-002}

The above figure shows the behaviour of the respective fund performance, which is \textbf{upward} trending for all the funds till the period of \textbf{"January-2008"}.For comparitive purpose, one can observe the distinct \textbf{drawdown} of \textbf{Newedge CTI} since the latter period.

\section{Statistical and Drawdown Analysis}

A summary of Fund Return series characteristics show that \textbf{DJUBS.Commodity} performs worse relatively to it's peers.The most distinct charactersitic being highest : \textbf{Variance, Stdev, SE Mean} and well as negative \textbf{Skewness} 

\begin{Schunk}
\begin{Soutput}
                DJUBS.Commodity Morningstar.CLS Newedge.CTI
Observations           151.0000        151.0000    151.0000
NAs                      0.0000          0.0000      0.0000
Minimum                 -0.2134         -0.1089     -0.0507
Quartile 1              -0.0251         -0.0142     -0.0058
Median                   0.0072          0.0055      0.0044
Arithmetic Mean          0.0024          0.0055      0.0103
Geometric Mean           0.0012          0.0049      0.0099
Quartile 3               0.0334          0.0250      0.0224
Maximum                  0.1299          0.1162      0.1263
SE Mean                  0.0041          0.0029      0.0023
LCL Mean (0.95)         -0.0056         -0.0002      0.0057
UCL Mean (0.95)          0.0104          0.0113      0.0149
Variance                 0.0025          0.0013      0.0008
Stdev                    0.0499          0.0358      0.0286
Skewness                -0.6378          0.1021      1.1051
Kurtosis                 1.9837          1.2806      2.2006
\end{Soutput}
\end{Schunk}


The results are consistent with Drawdown Analysis in which \textbf{DJUBS.Commodity} performs worse relatively to it's peers.

\begin{Schunk}
\begin{Soutput}
                             DJUBS.Commodity Morningstar.CLS Newedge.CTI
Semi Deviation                        0.0375          0.0249      0.0169
Gain Deviation                        0.0279          0.0250      0.0256
Loss Deviation                        0.0373          0.0229      0.0125
Downside Deviation (MAR=10%)          0.0406          0.0265      0.0158
Downside Deviation (Rf=0%)            0.0363          0.0221      0.0114
Downside Deviation (0%)               0.0363          0.0221      0.0114
Maximum Drawdown                      0.5452          0.2403      0.1162
Historical VaR (95%)                 -0.0714         -0.0477     -0.0272
Historical ES (95%)                  -0.1139         -0.0727     -0.0377
Modified VaR (95%)                   -0.0860         -0.0512     -0.0257
Modified ES (95%)                    -0.1337         -0.0706     -0.0360
\end{Soutput}
\end{Schunk}
\section{Non-i.i.d GSoC Usage}
\subsection{Auctocorrelation Adjusted Standard Devitation}
Given a sample of historical returns \((R_1,R_2, . . .,R_T)\),the method assumes the fund manager smooths returns in the following manner, when 't' is the unit time interval, with  $\rho$\ as the respective term autocorrelation coefficient

%Let $X \sim N(0,1)$ and $Y \sim \textrm{Exponential}(\mu)$.  Let
%$Z = \sin(X)$. $\sqrt{X}$.
  
%$\hat{\mu}$ = $\displaystyle\frac{22}{7}$
%e^{2 \mu} = 1
%\begin{equation}
%\left(\sum_{t=1}^{T} R_t/T\right) = \hat{\mu} \\
%\end{equation}
\begin{equation}
 \sigma_{T}  =   \sqrt{  \sum_k^n(\sigma_{t}^2 +  2*\rho_i) } \\
\end{equation}


\includegraphics{CommodityReport-005}

From the above figure, we can observe that all the funds, exhibit \textbf{serial auto correlation}, which results in significantly \emph{inflated} standard deveation.
\subsection{Andrew Lo Statistics of Sharpe Ratio}

The building blocks of the \textbf{Sharpe Ratio} : expected returns and volatilities  are unknown quantities that must be estimated statistically and are,
therefore, subject to \emph{estimation error} .To address this question, Andrew Lo derives explicit expressions for the statistical distribution of the Sharpe ratio using
standard asymptotic theory. 
Given a predefined benchmark ,\eqn{\hat{SR}}  can be expressed in terms of autocorrelated coefficients as
 
 \deqn{ \hat{SR} (q) - SR(q)= Normal Distribution(0,V_{GMM}(q)) }
 
In given commodity funds, we find results, similar reported in paper, that the annual Sharpe ratio for a hedge fund can be overstated by as much as \textbf{65} \% because of the presence of \textbf{serial correlation}.We can observe that the fund "\textbf{DJUBS.Commodity}", which has the largest drawdown and serial autocorrelation, has it's Andrew Lo Sharpe ratio , \emph{decrease} most significantly as comapared to other funds.

\includegraphics{CommodityReport-006}
\subsection{Conditional Drawdown}
A new one-parameter family of risk measures called Conditional Drawdown (CDD) has
been proposed. These measures of risk are functionals of the portfolio drawdown (underwater) curve considered in active portfolio management. For some value of $\hat{\alpha}$ the tolerance parameter, in the case of a single sample path, drawdown functional is defined as the mean of the worst (1 \(-\) $\hat{\alpha}$)100\% drawdowns. The CDD measure generalizes the notion of the drawdown functional to a multi-scenario case and can be considered as a generalization of deviation measure to a dynamic case. The CDD measure includes the Maximal Drawdown and Average Drawdown as its limiting cases.Similar to other cases, \textbf{DJUBS.Commodity}, is the worst performing fund with worst case conditional drawdown greater than \textbf{50\%} and \textbf{Newedge.CTI} performing significantly well among the peer commodity indices with less than \textbf{15\%}.

\includegraphics{CommodityReport-007}
\subsection{Calmar and Sterling Ratio}
Both the Calmar and the Sterling ratio are the ratio of annualized returnmover the absolute value of the maximum drawdown of an investment.
{equation}
\begin{equation}
 Calmar Ratio  =  \frac{Return [0,T]}{max Drawdown  [0,T]} \\
\end{equation}

\begin{equation}
 Sterling Ratio  =  \frac{Return [0,T]}{max Drawdown  [0,T] - 10\%} \\
\end{equation}
\begin{Schunk}
\begin{Sinput}
> CalmarRatio.Norm(COM.09,1)
\end{Sinput}
\begin{Soutput}
                        DJUBS.Commodity Morningstar.CLS Newedge.CTI
Normalized Calmar Ratio    0.0006270255      -0.2584815   0.2052236
\end{Soutput}
\begin{Sinput}
> SterlingRatio.Norm(COM.09,1)
\end{Sinput}
\begin{Soutput}
                                         DJUBS.Commodity Morningstar.CLS
Normalized Sterling Ratio (Excess = 10%)    0.0005298375       -0.182521
                                         Newedge.CTI
Normalized Sterling Ratio (Excess = 10%)   0.1102788
\end{Soutput}
\end{Schunk}
For a 1 year \emph{horizon} return, we can see that Newedge.CTI is the clear performer in this metric as well.However, a \textbf{suprising} observed result, is negative \emph{Sterling} and \emph{Calmar} ratio for Morningstar.CLS . 
\subsection{GLM Smooth Index}
A useful summary statistic for measuringthe concentration of weights is :
\begin{equation}
\xi =   \sum_{j=0}^{k} \theta _j^2 \\
\end{equation}

This measure is well known in the industrial organization literature as the Herfindahl index, a measure of the concentration of firms in a given industry where $\theta$\(_j\) represents the market share of firm j. Becaus $\xi_t$\ is confined to the unit interval, and is minimized when all the $\theta$\(_j\) 's are identical, which implies a value of 1/k+1 for $\xi_i$\ ; and is maximized when one coefficient is 1 and the rest are 0. In the context of smoothed returns, a lower value of implies more smoothing, and the upper bound of 1 implies no smoothing, hence we shall refer to $\theta$\(_j\) as a \textbf{smoothing index}.

\includegraphics{CommodityReport-009}

For the given chart, we can observe that \textbf{all the funds} have significant level of smooth returns.
\subsection{Acar Shane Maximum Loss}
The main concern of this paper is the study of alternative risk measures: namely maximum loss and 
maximum drawdown. Unfortunately, there is no analytical formulae to establish the maximum drawdown properties under the random walk assumption. We should note first that due to its definition, the maximum drawdown divided by volatility is an only function of the ratio mean divided by volatility. 
\begin{equation}
MD / \sigma =  Min \frac{ \sum_{j=1}^{t} X_{j}}{\sigma} = F(\frac{\mu}{\sigma}) \\
\end{equation}

Such a ratio is useful in that this is a complementary statistic to the return divided by volatility ratio. To get some insight on the relationships between maximum drawdown per unit of volatility and mean  return divided by volatility, we have proceeded to Monte-Carlo simulations. We have simulated cash flows over a period of 36 monthly returns and measured maximum drawdown for varied levels of  annualised return divided by volatility varying from minus two to two by step of 0.1. The process has  been repeated six thousand times.


\includegraphics{CommodityReport-010}

As we can see from the \emph{simulted chart}, DJUBS.Commodity comes at the bottom , which imply a \emph{lower} \textbf{return-maximum loss} ratio.

\includegraphics{CommodityReport-011}

Finally, from the autocorrelation lag plot, one can observe, significant \textbf{positive} autocorrelation for \textbf{Newedge.CTI}, which is a \emph{warning} signal in case drawdown occurs, in an otherwise excellent performing fund.
\section{Conclusion}

Analyzing all the function results, one can clearly differentiate \textbf{Newedge.CTI}, as a far superior fund as compared to it's peer.\textbf{MorningStar.CLS}, exbhibits highest autocorrelation as well as lowest Calmar/Sterling ratio, but compared on other front, it distincly outperforms \textbf{DJUBS.Commodity}, which has performed poorfly on all the tests. 

The above figure shows the characteristic of the respective fund performance, which is after the period of analysis till \textbf{"July-2013"}.At this moment, we would like the readers, to use the functions developed in the R \textbf{"PerformanceAnalytics"} package, to study ,use it for analysis as well as for forming their own opinion. 

\includegraphics{CommodityReport-012}


\end{document}
